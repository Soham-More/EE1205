% \iffalse
\let\negmedspace\undefined
\let\negthickspace\undefined
\documentclass[journal,12pt,twocolumn]{IEEEtran}
\usepackage{cite}
\usepackage{amsmath,amssymb,amsfonts,amsthm}
\usepackage{algorithmic}
\usepackage{graphicx}
\usepackage{textcomp}
\usepackage{xcolor}
\usepackage{txfonts}
\usepackage{listings}
\usepackage{enumitem}
\usepackage{mathtools}
\usepackage{gensymb}
\usepackage{comment}
\usepackage[breaklinks=true]{hyperref}
\usepackage{tkz-euclide} 
\usepackage{listings}
\usepackage{gvv}                                        
\def\inputGnumericTable{}                                 
\usepackage[latin1]{inputenc}                                
\usepackage{color}                                            
\usepackage{array}                                            
\usepackage{longtable}                                       
\usepackage{calc}                                             
\usepackage{multirow}                                         
\usepackage{hhline}                                           
\usepackage{ifthen}                                           
\usepackage{lscape}
\usepackage[center]{caption} % center the captions to figure

\newtheorem{theorem}{Theorem}[section]
\newtheorem{problem}{Problem}
\newtheorem{proposition}{Proposition}[section]
\newtheorem{lemma}{Lemma}[section]
\newtheorem{corollary}[theorem]{Corollary}
\newtheorem{example}{Example}[section]
\newtheorem{definition}[problem]{Definition}
\newcommand{\BEQA}{\begin{eqnarray}}
\newcommand{\EEQA}{\end{eqnarray}}
\newcommand{\define}{\stackrel{\triangle}{=}}
\theoremstyle{remark}
\newtheorem{rem}{Remark}
\begin{document}

\newcolumntype{M}[1]{>{\centering\arraybackslash}m{#1}}
\newcolumntype{N}{@{}m{0pt}@{}}

\bibliographystyle{IEEEtran}
\vspace{3cm}

\title{GATE 2021 ME 3Q} 
\author{ee23btech11223 - Soham Prabhakar More% <-this % stops a space
}
\maketitle
\newpage
\bigskip

\renewcommand{\thefigure}{\theenumi}
\renewcommand{\thetable}{\theenumi}

\bibliographystyle{IEEEtran}

\textbf{Question:} The Dirac-delta function $\brak{\delta\brak{t - t_0}}$ for $t, t_0 \in \Re$, has the following property
\begin{align}
    \int_{a}^{b}\phi\brak{t}\delta\brak{t - t_0}dt = 
    \begin{cases}
        \phi\brak{t_0}\quad a < t_0 < b\\
        0 \quad\quad otherwise
    \end{cases} \label{eq:2022.ME.3.1}
\end{align}

The Laplace Transform of the Dirac-delta function $\delta\brak{t - a}$ for $a > 0; \mathcal{L}\brak{\delta\brak{t - a}} = F\brak{s}$ is

\hfill{(GATE 2021 ME 3Q)}

\solution
\begin{table}[ht]
    \begin{table}[ht]
\renewcommand\thetable{1}
\begin{tabular}{|c|c|c|}
    \hline 
    \textbf{Parameter}&\textbf{Description} &\textbf{Value}\\
    \hline 
    $r_i$ & Common ratio of G.P (a),(b),(c) & $\sqrt{2}, \sqrt{3}, \frac{1}{3}$ \\
    \hline 
    $x_i\brak{n}$ & Sequence & $x_i\brak{0}r_i^nu[n]$ \\
    \hline 
	$X_i\brak{z}$ & Transform of $x_i\brak{n}$ & $\frac{x\brak{0}}{1-rz^{-1}}$ \\
    \hline
\end{tabular}

\caption{Table of parameters}
\label{Table:1}
\end{table}

\end{table} \\

By \eqref{eq:2022.ME.3.1} and $a > 0$,
\begin{align}
    F\brak{s} &= \int_{0}^{\infty}\delta\brak{t - a}e^{-st}dt \\
    \therefore F\brak{s} &= e^{-as}
\end{align}

The fourier transform,
\begin{align}
    G\brak{f} &= \int_{-\infty}^{\infty}\delta\brak{t - a}e^{-2\pi jft}dt \\
    \therefore G\brak{f} &= e^{-j2\pi fa}
\end{align}
For a periodic signal the fourier transform is defined as:
\begin{align}
    H\brak{f} &= \sum_{k = -\infty}^{\infty}c_k\delta\brak{f - \frac{k}{T}}
\end{align}
where $c_k$ are the fourier series coefficients and $T$ is the period. Thus,
\begin{align}
    W_T\brak{f} &= \sum_{k = -\infty}^{\infty}c_k\delta\brak{f - \frac{k}{T}} \\
    c_k &= \frac{1}{T}\int_{-\frac{T}{2}}^{\frac{T}{2}}w_T\brak{t}e^{-j2\pi \frac{k}{T}f}dt \\
    c_k &= \frac{1}{T}\int_{-\frac{T}{2}}^{\frac{T}{2}}\brak{\sum_{k = -\infty}^{\infty}\delta\brak{t - kT}}e^{-j2\pi \frac{k}{T}f}dt \\
\end{align}
\begin{align}
    c_k &= \frac{1}{T}\sum_{k = -\infty}^{\infty}\int_{-\frac{T}{2}}^{\frac{T}{2}}\delta\brak{t - kT}e^{-j2\pi \frac{k}{T}f}dt \\
    c_k &= \frac{1}{T} \\
    W_T\brak{f} &= \frac{1}{T}\sum_{k = -\infty}^{\infty}\delta\brak{f - \frac{k}{T}} \\
    \therefore W_T\brak{f} &= \frac{1}{T}w_{\frac{1}{T}}\brak{f}
\end{align}
Thus, the fourier transform of impulse train is another impulse train.
%\begin{align}
%    f\brak{t} &\system{F} H\brak{f} \\
%    f\brak{t + T} &\system{F} e^{j2\pi fT}H\brak{f} \\
%    \because e^{j2\pi fT}H\brak{f} &= H\brak{f} \\
%    H\brak{f}\brak{1 - e^{j2\pi fT}} &= 0
%\end{align}
%Thus, $H\brak{f}$ is zero everywhere except at $f = \frac{n}{T}, n \in Z$
%\begin{align}
%    \therefore H\brak{f} &= \sum_{k = -\infty}^{\infty}c_k\delta\brak{f - \frac{k}{T}} \\
%    \because \sum_{k = -\infty}^{\infty}c_ke^{-j2\pi f\frac{k}{T}} &\system{F} H\brak{f}
%\end{align}
%$c_k$ are the fourier series coefficents of $h\brak{t}$,
%\begin{align}
%    c_k = \int_{-\frac{T}{2}}^{\frac{T}{2}}
%\end{align}
\end{document}
